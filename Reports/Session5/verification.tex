\section{System Verification}
The aim of this chapter is to verify the righteousness of the model by putting the translation of the requirements into mcf file code. This .mcf file is then run over the mCRL2 model and should return true for every requirement if it satisfies these requirements. The modal {\textmu}-calculus formulas are found below.
\subsection{Formulas}
\subsubsection{Only one gate can be open at a time}
\textit{It is only possible to send an openGate(gateID$_i$) to open gate $i$ after gateState$_j$($ b $) delivers $b = false$ and openGate(gateID$_j$) has not been issued in the meantime, for gateID$_i \neq$ gateID$_j$:}
	\begin{itemize}
		\item ~[$\overline{\textrm{gateState$_j$(false)}}$*.openGate(gateID$_i$)]false
		\item ~[true*.openGate(gateID$_j$).$\overline{\textrm{gateState$_j$(false)}}$*.openGate(gateID$_i$)]false
	\end{itemize}	
	
\subsubsection{Only one valve can be open at a time}
\textit{It is only possible to send an openValve(valveID$_i$) to open valve $i$ after \linebreak valveState$_j$($ b $) delivers $b = false$ and openValve(valveID$_j$) has not been issued in the meantime, for valveID$_i \neq$ valveID$_j$:}
	\begin{itemize}
		\item ~[$\overline{\textrm{valveState$_j$(false)}}$*.openValve(valveID$_i$)]false
		\item ~[true*.openValve(valveID$_j$).$\overline{\textrm{valveState$_j$(false)}}$*.openValve(valveID$_i$)]false
	\end{itemize}
	
\subsubsection{A valve and a gate at different levels can never be open at the same time}
\textit{It is only possible to send an openGate(gateID$_i$) to open gate $i$ after valveState$_j$($ b $) delivers $b = false$ and openValve(valveID$_j$) has not been issued in the meantime, for gate $i$ and valve $j$ are on opposite sides of the lift:}
	\begin{itemize}
		\item ~[$\overline{\textrm{valveState$_j$(false)}}$*.openGate(gateID$_i$)]false
		\item ~[true*.openValve(valveID$_j$).$\overline{\textrm{valveState$_j$(false)}}$*.openGate(gateID$_i$)]false
	\end{itemize}

\noindent\textit{It is only possible to send an openValve(valveID$_i$) to open valve $i$ after gateState$_j$($ b $) delivers $b = false$ and openGate(gateID$_j$) has not been issued in the meantime, for valve $i$ and gate $j$ are on opposite sides of the lift:}
	\begin{itemize}
		\item ~[$\overline{\textrm{gateState$_j$(false)}}$*.openValve(valveID$_i$)]false
		\item ~[true*.openGate(gateID$_j$).$\overline{\textrm{gateState$_j$(false)}}$*.openValve(valveID$_i$)]false
	\end{itemize}
	
\subsubsection{A gate can only close if there is no ship in between its doors}
\textit{It is only possible to send closeGate(gateID$_i$) to close gate $i$ after gateSensor$_i$($b$) delivers $b = false$ and no other action has been issued in the meantime:}
	\begin{itemize}
		\item ~[true*.$\overline{\textrm{gateSensor$_i$(false)}}$.closeGate(gateID$_i$)]false
	\end{itemize}
	
\subsubsection{A gate can only open if the water level on both sides is equal}
\textit{It is only possible to send openGate(gateID$_i$) to open gate $i$ after \linebreak compareWaterLevel$_j$($b$) delivers $b = true$ and no other action has been issued in the meantime, for gate $i$ and container $j$ are on the same side of the lift:}
	\begin{itemize}
		\item $\overline{\textrm{compareWaterLevel$_j$(true)}}$.openGate(gateID$_i$)]false
		\item ~[true*.openValve(valveID$_j$).$\overline{\textrm{compareWaterLevel$_j$(true)}}$.openGate(gateID$_i$)]false
	\end{itemize}

\subsubsection{Signal lights can never be off nor indicate hold and pass at the same time}
\textit{passSignalOn(signalID$_i$) has to be followed directly by haltSignalOff(signalID$_i$):}
	\begin{itemize}
		\item ~[true*.passSignalOn(signalID$_i$).$\overline{\textrm{haltSignalOff(signalID$_i$)}}$]false
	\end{itemize}
\textit{haltSignalOn(signalID$_i$) has to be followed directly by passSignalOff(signalID$_i$):}
	\begin{itemize}
		\item ~[true*.haltSignalOn(signalID$_i$).$\overline{\textrm{passSignalOff(signalID$_i$)}}$]false
	\end{itemize}
\textit{passSignalOff(signalID$_i$) can only be issued directly after haltSignalOn(signalID$_i$):}
	\begin{itemize}
		\item ~[true*.$\overline{\textrm{haltSignalOn(signalID$_i$)}}$.passSignalOff(signalID$_i$)]false
	\end{itemize}
\textit{haltSignalOff(signalID$_i$) can only be issued directly after passSignalOn(signalID$_i$):}
	\begin{itemize}
		\item ~[true*.$\overline{\textrm{passSignalOn(signalID$_i$)}}$.haltSignalOff(signalID$_i$)]false
	\end{itemize}

\subsubsection{Signal lights can only indicate pass if the gate is completely opened}
\textit{It is only possible to send passSignalOn(signalID$_i$) to set signal $i$ to pass after a gateState$_j$($ b $) delivers $b = true$ and closeGate(gateID$_j$) has not been issued in the meantime, for signal $i$ and gate $j$ are on the same side of the lift:}
	\begin{itemize}
		\item ~[$\overline{\textrm{gateState$_j$(true)}}$*.passSignalOn(signalID$_i$)]false
		\item ~[true*.closeGate(gateID$_j$).$\overline{\textrm{gateState$_j$(true)}}$*.passSignalOn(signalID$_i$)]false
	\end{itemize}
	
\subsubsection{An entering signal can only indicate pass if there is no ship in the lift already}
\textit{It is only possible to send passSignalOn(signalID$_i$) to set the entering signal $i$ to pass after boatExitUp or boatExitDown and neither goToPos00 nor goToPos10 have been issued in the meantime:}\vspace{0.3cm}

	$\bullet$ ~[true*.(goToPos00 $||$ goToPos10).$\overline{\textrm{(boatExitUp} || \textrm{boatExitDown)}}$*.
		\begin{flushright}
			passSignalOn(SignalID$_i$)]false
		\end{flushright}
		
\subsection{Functionality Formulas}
\textbf{After a ship entered the system, this ship should be transported to the other level before any other ship is allowed to enter the system}
\textit{After a goToPos01 action, a boatExitUp should be issued and not an goToPos11 action between.}

\begin{itemize}
	\item ~[true*.goToPos01.$\overline{\textrm{boatExitUp}}$*.(goToPos01 $\vee$ goToPos11)]false
\end{itemize}

\noindent\textit{After a goToPos11 action, a boatExitDown should be issued and not an goToPos01 action between.}

\begin{itemize}
	\item ~[true*.goToPos11.$\overline{\textrm{boatExitDown}}$*.(goToPos01 $\vee$ goToPos11)]false
\end{itemize}

\subsection{Verification steps}
To verify if the mCRL2 model satisfies these requirements a series of steps need to be taken. First of all while running the mCRL2 GUI the model needs to be transformed into a linear process specification by doing a mcrl22lps transformation. Assuming the formulas are already put in a .mcf file the model should be then transformed using lps2pbes. Within the option menu of this transformation the .mcf file can then be added at formulas. After this transformation the requirements can then be verified by running pbes2bool.
\pagebreak
\subsection{Results}
As the table below shows every evaluated requirement returns true indicating that the model of the boat lift system satisfies the requirements:

\begin{table}[htbp]
	\centering
	\begin{tabular}{ll}
		\toprule
		\textbf{Requirement} &  \textbf{Result} \\
		\hline
		8.1.1&  True \\
		8.1.2&  True \\
		8.1.3&  True \\
		8.1.4&  True \\
		8.1.5&  True \\
		8.1.6&  True \\
		8.1.7&  True \\
		8.1.8&  True \\
		\bottomrule
	\end{tabular}%
	\caption{Verification results}
	\label{tab:sort}%
\end{table}%
%--------------------------------------------------------------------------------------------------------------------------------------------%
%																FUNCTIONALITY
%--------------------------------------------------------------------------------------------------------------------------------------------%
\begin{comment}
\subsection{Functionality}
\subsubsection{If a ship has just left the system and there is no other boat waiting, the system should enter idle state:}
\textit{If shipPresence(posID$_i$, $b$) returns $b = false$ for all position IDs posID$_i$:}
	\begin{itemize}
		\item \textit{For all gate IDs gateID$_i$: If gateState(gateID$_i$, $b$) returns $b = true$, \linebreak closeGate(gateID$_i$) must be issued}
		\begin{itemize}
			\item ~[true*.gateState(gateID$_i$, true).$\overline{\textrm{closeGate(gateID$_i$)}}$]false
		\end{itemize}
		\item \textit{For all signal IDs signalID$_i$: If signalState(signalID$_i$, $b$) returns $b = true$, haltSignal(signalID$_i$, true) must be issued}
		\begin{itemize}
			\item ~[true*.signalState(signalID$_i$, true).$\overline{\textrm{haltSignal(signalID$_i$, true)}}$]false
		\end{itemize}
		\item \textit{Otherwise only shipPresence(posID$_i$, $b$) action must happen}
		\begin{itemize}
			\item ~[true*.$\overline{\textrm{shipPresence(posID$_i$, b)}}$]false
		\end{itemize}
	\end{itemize}
	
\subsubsection{If there is no ship in the lift the first ship to arrive in the system should get served first}
\textit{While the system is idle, it should check for arriving ships periodically using the shipPresence(posID$_i$, $b$) action for all position IDs posID$_i$. If $b = true$ for one position posID$_i \neq X.1$, the system should issue passSignal(signalID$_i$, true) where signalID$_i$ is the entering signal light at the detected ship's position:}\vspace{0.3cm}

		% [true*.a]uX.([!b]X && <true>true) -> after an a, b will always happen at some point:
		$ \bullet $ [true*.shipPresence(posID$_i$, true)]$ \mu X$.(
			\begin{flushright}
				~[$ \overline{\textrm{passSignal(signalID$_i$, true)}} ]X \wedge <$true$>$true)
			\end{flushright}
	
\subsubsection{If a ship has just left the system and there are boats waiting on one side of the lift, the first boat in line gets served}
\textit{If shipPresence(posID$_i$, $b$) returns $b = true$ for one position posID$_i \neq X.1$, the system should issue passSignal(signalID$_i$, true) where signalID$_i$ is the entering signal light at position posID$_i$:}\vspace{0.3cm}

		$ \bullet $ [true*.shipPresence(posID$_i$, true)]$ \mu X$.(
		\begin{flushright}
			~[$ \overline{\textrm{passSignal(signalID$_i$, true)}} ]X \wedge <$true$>$true)
		\end{flushright}
	
\subsubsection{If a ship has just left the system and there are boats waiting on both sides of the lift, then the first boat on the current level gets served}
\textit{If both, shipPresence(posID$_i$, $b$) and shipPresence(posID$_j$, $b$) return $b = true$ for posID$_i \neq$ posID$_j \neq$ posID$_{X.1}$, the system should check the current waterlevel in container X.1 using compareWaterLevel(containerID$_i$, $b_i$) and \linebreak compareWaterLevel(containerID$_j$, $b_j$) (containerID$_i \neq$ containerID$_j \neq$ \linebreak containerID$_{X.1}$) and then issue passSignal(signalID$_k$, true) where signalID$_k$ is the entering signal light at container $k$ and
		\begin{equation*}
		k = 
		\begin{cases}
		i & \text{if } b_i = true\\
		j & \text{if } b_j = true\\
		\end{cases}
		\end{equation*}
}\vspace{0.3cm}

$ \bullet $ [true*.shipPresence(posID$_i$, true)]$ \mu X$.(
\begin{flushright}
	~[$ \overline{\textrm{passSignal(signalID$_k$, true)}} ]X \wedge <$true$>$true)
\end{flushright}
	
\subsubsection{If a ship hast just entered the lift the system should transport the ship to the other level}
\textit{Once a ship has entered container X.1 (shipPresence(posID$_{X.1}$, $b$) results in $b = true$ for the first time after returning $b = false$), it should check the current waterlevel in container X.1 using compareWaterLevel(containerID$_i$, $b_i$) and \linebreak compareWaterLevel(containerID$_j$, $b_j$) (containerID$_i \neq$ containerID$_j \neq$ \linebreak containerID$_{X.1}$) and then issue passSignal(signalID$_k$, true) where signalID$_k$ is the leaving signal light at container $k$ and
		\begin{equation*}
		k = 
		\begin{cases}
		i & \text{if } b_i = true\\
		j & \text{if } b_j = true\\
		\end{cases}
		\end{equation*}
}\vspace{0.3cm}

$ \bullet $ [true*.shipPresence(posID$_{X.1}$, true)]$ \mu X$.(
\begin{flushright}
	~[$ \overline{\textrm{passSignal(signalID$_k$, true)}} ]X \wedge <$true$>$true)
\end{flushright}}
\end{comment}