\section{mCRL System Behaviour Modelling}
This section is about the the model of the ship lift system in MCRL2 code. In appendix A it is shown the complete code for any clarification.
\subsection{The code}
The code has been divided into the 4 controllers: 1 main controller and 3 sub controllers namely gate, valve and signal controller. The communication between each controller to the main one and vice versa are clarified in Figure \ref{fig:subcontrollers}).
It has been created a series of sort element has it is shown in the following table \ref{tab:sort}.
			\begin{table}[htbp]
	\centering
	\caption{)}
	\begin{tabular}{lp{6cm}l}
		\toprule
		\textbf{Sort} &  \textbf{Description} \\
		\hline
		State&  define states of the main controller \\
		M2G &  define communication element from master to gate controller\\
		G2M & define communication element from gate to main controller \\
		M2S &  define communication element from master to signal controller\\
		S2M &  define communication element from signal to master controller\\
		M2V &  define communication element from master to valve controller\\
		V2M &  define communication element from valve to master controller\\
		SLID &  define signal light ID elemnt\\
		GateID &  define gate ID elemnt\\
		ValveID &  define valve ID elemnt\\
		PositionID   & define position ID element\\
		
	\end{tabular}%
	\label{tab:sort}%
\end{table}%
The main controller has been divided into a series of states. Each states has only one action and then it goes to an other state or create a loop to himself. This loop-mechanism has been created for let the main controller being able to receive proper information about the status of the systems from the other sub-controllers. The initial state of the boat lift has been defined by making the master controller to idle state.\\
On MCRL2, in the graph and in the view tools it is possible to analyze the system. It is mainly composed by two main branch that start from the idle state, they describe the lift when the boat is going up and on the other branch when the ship is going down. The system is then able to manage one operation at the time and for change branch it is always needed to go to the idle state. This will be very helpful for fulfill all the requirement that are going to be checked on the next section of this paper. Moreover, in the tools is it possible to look that many states has some loops needed for two reasons: for the sub-controllers to check all the boat lift state and for the master for check for response from valve, gate and signal sub controllers.

